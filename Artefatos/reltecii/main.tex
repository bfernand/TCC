%% Documento gerado após muito esforço por Bruno Fernandes

\documentclass[
    % -- opções da classe memoir --
    12pt,               % tamanho da fonte
    openright,          % capítulos começam em pág ímpar (insere página vazia caso preciso)
    oneside,            % para impressão em verso e anverso. Oposto a % oneside
    a4paper,            % tamanho do papel. 
    % -- opções da classe abntex2 --
    chapter=TITLE,     % títulos de capítulos convertidos em letras maiúsculas
    %section=TITLE,     % títulos de seções convertidos em letras maiúsculas
    %subsection=TITLE,  % títulos de subseções convertidos em letras maiúsculas
    %subsubsection=TITLE,% títulos de subsubseções convertidos em letras maiúsculas
    %subsubsubsection=TITLE,% títulos de subsubsubseções convertidos em letras maiúsculas
    % -- opções do pacote babel --
    english,            % idioma adicional para hifenização
    spanish,            % idioma adicional para hifenização
    portuguese              % o último idioma é o principal do documento
    ]{abntex2} 

% ---
% Pacotes básicos 
% ---
\label{preâmbulo}

\usepackage{lmodern}            % Usa a fonte Latin Modern          
\usepackage[T1]{fontenc}        % Selecao de codigos de fonte.
\usepackage[utf8]{inputenc}     % Codificacao do documento (conversão automática dos acentos)
\usepackage{lastpage}           % Usado pela Ficha catalográfica
\usepackage{indentfirst}        % Indenta o primeiro parágrafo de cada seção.
\usepackage{color,xcolor}       % Controle das cores
\usepackage{graphicx}           % Inclusão de gráficos
\usepackage{microtype}          % para melhorias de justificação
\usepackage{booktabs}
\usepackage{tabulary}
\usepackage{caption}
\usepackage{soul}
\usepackage{rotating}

\renewenvironment{quotation}%
  {%
   \small
   \begin{list}{}{%
       \setlength{\listparindent}{0cm}%
       \setlength{\itemindent}{\listparindent}%
       \setlength{\rightmargin}{0cm}%
       \setlength{\leftmargin}{4cm}%
       \setlength{\parsep}{0pt}}%
    \item\relax}%
  {\end{list}}

\let\newfloat\undefined
\usepackage{floatrow}

\floatsetup[table]{capposition=top}
\floatsetup[figure]{capposition=top}

\captionsetup[table]{justification=centering,width=\textwidth,labelfont=bf,textfont=bf}
\captionsetup[lstlisting]{justification=centering,width=\textwidth,labelfont=bf,textfont=bf}
\captionsetup[figure]{justification=centering,width=\textwidth,labelfont=bf,textfont=bf}

	\usepackage{helvet}
	\renewcommand{\familydefault}{\sfdefault}

% ---



% ---
% Pacotes de citações
% ---

\usepackage[brazilian,hyperpageref]{backref}     % Paginas com as citações na bibl
\usepackage[alf]{abntex2cite}   % Citações padrão ABNT

% ---
% Redefinição do comando duplo quote e quote simples
% ---
\newcommand\dblquote[1]{\textquotedblleft #1\textquotedblright}
\newcommand\sglquote[1]{\textquoteleft #1\textquoteright}

% --- 
% CONFIGURAÇÕES DE PACOTES
% --- 

% ---
% Configurações do pacote backref
% Usado sem a opção hyperpageref de backref
\renewcommand{\backrefpagesname}{Citado na(s) página(s):~}
% Texto padrão antes do número das páginas
\renewcommand{\backref}{}
% Define os textos da citação
\renewcommand*{\backrefalt}[4]{
    \ifcase #1 %
        Nenhuma citação no texto.%
    \or
        Citado na página #2.%
    \else
        Citado #1 vezes nas páginas #2.%
    \fi}%
% ---


% ---
% Configurações de aparência do PDF final

% alterando o aspecto da cor azul
\definecolor{blue}{RGB}{41,5,195}

% informações do PDF
\makeatletter
\hypersetup{
        %pagebackref=true,
        pdftitle={\@title}, 
        pdfauthor={\@author},
        pdfsubject={\imprimirpreambulo},
        pdfcreator={LaTeX with abnTeX2},
        pdfkeywords={fastformat.co}, 
        %
        colorlinks=true,            % false: boxed links; true: colored links
        linkcolor=blue,             % color of internal links
        citecolor=blue,             % color of links to bibliography
        filecolor=magenta,          % color of file links
        urlcolor=blue,
        %
        bookmarksdepth=4
}
\makeatother
% --- 

% ---
% Informações de dados para CAPA e FOLHA DE ROSTO
% ---
\titulo{Estudo de como metodologias ágeis atendem boas práticas de gerenciamento de projetos de Software}

\autor{Bruno Fernandes}
\local{Maringá}
\data{Fevereiro de 2016}
\orientador{Prof. Dr. Donizete Bruzarosco}

\instituicao{
  \bfseries
  \MakeUppercase{Universidade Estadual de Maringá}
  \par
  \MakeUppercase{Departamento de Informática}
  \par
  \MakeUppercase{Bacharelado em Informática} }
\tipotrabalho{Trabalho de Conclusão de Curso}
% O preambulo deve conter o tipo do trabalho, o objetivo, 
% o nome da instituição e a área de concentração 
\preambulo{Monografia apresentada ao curso de Informática da UEM, como requisito para obtenção do título de bacharel em Informática.}
% ---

\def\volume{  }


% --- 
% Espaçamentos entre linhas e parágrafos 
% --- 

% O tamanho do parágrafo é dado por:
\setlength{\parindent}{1.3cm}

% Controle do espaçamento entre um parágrafo e outro:
\setlength{\parskip}{0.2cm}  % tente também \onelineskip

% ---
% compila o indice
% ---
\makeindex
% ---

% ----
% Início do documento
% ----
\label{inicio do doc}
\begin{document}


% Seleciona o idioma do documento (conforme pacotes do babel)
%\selectlanguage{english}
\selectlanguage{brazil}

% Retira espaço extra obsoleto entre as frases.
\frenchspacing 

% ---
\renewcommand{\ABNTEXchapterfontsize}{\LARGE}
\renewcommand{\ABNTEXpartfontsize}{\ABNTEXchapterfontsize}
\renewcommand{\ABNTEXsectionfontsize}{\Large}
\renewcommand{\ABNTEXsubsectionfontsize}{\large}
\renewcommand{\ABNTEXsubsubsectionfontsize}{\normalsize}
\renewcommand{\ABNTEXsubsubsubsectionfontsize}{\normalsize}
% ---

% ----------------------------------------------------------
% ELEMENTOS PRÉ-TEXTUAIS
% ----------------------------------------------------------
% \pretextual

% ---
% Capa
% ---
\label{capa}
\imprimircapa

% ---

% ---
% Folha de rosto
% (o * indica que haverá a ficha catalográfica)
% ---
%\label{folha rosto}
%\imprimirfolhaderosto*
% ---

% ---
% Inserir a ficha catalográfica
% ---
% ---

% ---
% Inserir errata
% ---
\label{errata}
% \begin{errata}
% Elemento opcional da \citeonline[4.2.1.2]{NBR14724:2011}. Exemplo:
% 
% \vspace{\onelineskip}
% 
% FERRIGNO, C. R. A. \textbf{Tratamento de neoplasias ósseas apendiculares com
% reimplantação de enxerto ósseo autólogo autoclavado associado ao plasma
% rico em plaquetas}: estudo crítico na cirurgia de preservação de membro em
% cães. 2011. 128 f. Tese (Livre-Docência) - Faculdade de Medicina Veterinária e
% Zootecnia, Universidade de São Paulo, São Paulo, 2011.
% 
% \begin{table}[htb]
% \center
% \footnotesize
% \begin{tabular}{|p{1.4cm}|p{1cm}|p{3cm}|p{3cm}|}
%   \hline
%    \textbf{Folha} & \textbf{Linha}  & \textbf{Onde se lê}  & \textbf{Leia-se}  \\
%     \hline
%     1 & 10 & auto-conclavo & autoconclavo\\
%    \hline
% \end{tabular}
% \end{table}
% 
% \end{errata}
% ---

% ---
% Inserir folha de aprovação
% Substituir por versão digitalizada ao finalizar o TCC
% ---
% ---

% ---
% Dedicatória
% ---

% ---

% ---
% Agradecimentos
% --

% ---

% ---
% Epígrafe
% ---

% ---

% ---
% RESUMOS
% ---
% resumo em português


% resumo em inglês


% resumo em francês 


% resumo em espanhol

% ---

% ---
% inserir lista de ilustrações
% ---
\label{lista ilustrações}
%  \pdfbookmark[0]{\listfigurename}{lof}
%  \listoffigures*
%  \cleardoublepage

% ---

% ---
% inserir lista de tabelas
% ---
\label{lista tabelas}
%  \pdfbookmark[0]{\listtablename}{lot}
%  \listoftables*
%  \cleardoublepage

% ---

% ---
% inserir lista de abreviaturas e siglas
% ---
\label{lista siglas}
%\begin{siglas}
%	\item[PMBOK] Project Management Body of Knowledge
%	\item[PMI] Project Management Institute
%	\item[ABNT] Associação Brasileira de Normas Técnicas
%\end{siglas}

% ---
% inserir lista de símbolos
% ---
% \begin{simbolos}
%   \item[$ \Gamma $] Letra grega Gama
%   \item[$ \Lambda $] Lambda
%   \item[$ \zeta $] Letra grega minúscula zeta
%   \item[$ \in $] Pertence
% \end{simbolos}
% ---

% ---
% inserir o sumario
% ---
\tableofcontents
% ---

% ----------------------------------------------------------
% ELEMENTOS TEXTUAIS
% ----------------------------------------------------------
\textual


\chapter{Atividades Desenvolvidas}

No início foi definido o orientador que ajudaria na condução do presente trabalho. Após algumas reuniões, o tema foi ficando um pouco mais claro e foi iniciado o desenvolvimento do trabalho. Foi realizada uma pesquisa bibliográfica, levando em considerações autores com grande conhecimento de gerenciamento de projetos e de engenharia de software. Alguns autores foram considerados pelo grande conhecimento em gerência de projetos, mesmo não atuando na área de engenharia de software, como é o caso do \citeonline{kerzner2011} e do próprio \citeonline{pmi2013}. O projeto do TCC foi realizado e entregue ao orientador e aos membros da banca. A revisão bibliográfica ainda está sendo desenvolvida. 

\chapter{Resultado obtidos e dificuldades encontradas}
Uma das grandes dificuldades encontradas foi com relação à formatação do trabalho conforme as normas ABNT. No início, foi experimentada a ferramenta Microsoft Word como editor de texto padrão. Porém, com o uso desta ferramenta, foi percebido que estava sendo disperso muito tempo com a formatação e pouco tempo com o conteúdo do trabalho. Assim, foi avaliada a possibilidade e decidido que seria melhor usar uma ferramenta que apoiasse na formatação das regras ABNT. Com o uso do compilador \LaTeX ~e a classe ABNTEX2, sendo usados no editor Texmaker, foi obtida significativa melhora no tempo de formatações, citações e listagem de bibliografias.

\begin{figure}[htb]
\RawFloats
	\caption{Editor Texmaker para \LaTeX} \label{fig:texmaker}
	\begin{center}
	    \includegraphics[scale=0.35]{figuras/texmaker.png}
	\end{center}
\end{figure}

Outra dificuldade encontrada é com a revisão bibliográfica. O Assunto de gerenciamento de projetos é extenso, bem como o Guia PMBOK. Está sendo difícil selecionar partes que serão mais relevantes para a comparação com metodologias ágeis, já que o PMBOK é bem detalhado, contrastando com as metodologias ágeis que apontam uma ideia mas não explana de forma detalhada o \dblquote{como} fazer.

\chapter{Avaliação do cronograma de execução}

Considerando o cronograma inicial o projeto do trabalho está atrasado, pois deveriam estar sendo colhidos os resultados da comparação entre as metodologias. As atividades que ainda precisam ser realizadas são: a finalização da revisão bibliográfica, coleta dos dados e comparação entre metodologias ágeis e o guia PMBOK, conclusão da redação da monografia, revisão e entrega.

\chapter{Referencial Teórico}

\section{Conceitos Básicos}

\subsection{Gerenciamento de projetos de software}

De acordo com o \textit{Project Management Institute} \cite{pmi2013}, projeto\index{Projeto} é \dblquote{um esforço temporário empreendido para criar um produto, serviço ou resultado único}. Temporário porque um projeto precisa ter começo e fim definidos e único pois deve ser, de alguma forma, diferente de todos os produtos, serviços e resultados semelhantes. Adicionando-se à isto, um projeto possui limite de financiamento, ou orçamento, e consome recursos humanos e não humanos, ou seja, dinheiro, pessoas, máquinas, entre outros \cite[p.~2]{kerzner2011}. É importante salientar, também, o que não é um projeto. \dblquote{Projetos não devem ser confundidos com o trabalho diário. Um projeto não é rotineiro nem repetitivo} \cite[p.~6]{grayLarson2009}.

Segundo \citeonline[p.~5]{grayLarson2009}, o maior objetivo de um projeto de desenvolvimento de software, assim como a maioria dos esforços de uma organização, é a satisfação de um cliente. Mas existem 5 principais características de um projeto, que o diferencia de outros esforços da Organização: 

\begin{alineas}
	\item possui objetivo estabelecido;
	\item possui período de validade definido;
	\item geralmente conta com o envolvimento de diversos departamentos e profissionais;
	\item comumente é para a elaboração de algo nunca antes realizado;
	\item possui tempo, custo e requerimentos de desempenho específicos.
\end{alineas}

No Brasil existem dois termos parecidos, mas com sentidos diferentes e que não devem ser confundidos:

\begin{alineas}
	\item Projeto de Software;
	\item Projeto de desenvolvimento de Software.
\end{alineas}

O primeiro é o \textit{Software Design}, em inglês, ou seja, é um processo iterativo por meio do qual os requisitos são traduzidos em um \dblquote{documento} para construção do software. O segundo vem do inglês \textit{Project}, que é de fato o esforço para criação de um produto, serviço ou resultado único. Projeto (\textit{Project}) não está relacionado apenas a Softwares. Podem ser aplicados às várias áreas de conhecimento humano \cite{pressman2006}.

Para que um projeto obtenha sucesso é altamente recomendado que haja um acompanhamento, ou gerenciamento do projeto. Segundo o \citeonline{pmi2013}, Gerenciamento de Projetos é \dblquote{a aplicação de conhecimentos, habilidades, ferramentas e técnicas às atividades do projeto a fim de atender aos seus requisitos}. Gerenciamento de projetos também é um estilo de administração orientado a resultados que premia a criação de relacionamentos colaborativos entre as diferentes pessoas de uma equipe \cite[p.~3]{grayLarson2009}.

Para \citeonline[p.~11]{Cruz2013}, a obtenção do objetivo é alcançada quando o gerenciamento de projetos contempla pelo menos os seguintes itens:

\begin{alineas}
	\item identificação dos requisitos;
	\item adaptação às diferentes expectativas das partes e às mudanças ao longo do ciclo de vida; e
	\item balanceamento adequado às restrições do projeto (Escopo, Qualidade, Cronograma, Orçamento, Recursos e Riscos).
\end{alineas}

Apesar dos conceitos acima se aplicarem a projetos de diversas áreas, projetos de desenvolvimento de software possuem algumas características distintas de outros projetos que podem fazer com que esse seja particularmente desafiador. Segundo \citeonline[p.~60]{sommerville2011}, algumas dessas diferenças são:

\begin{alineas}
	\item O produto é intangível. O software não pode ser visto ou tocado, deste modo, não há como o gerente de projetos saber o progresso do projeto apenas olhando para o artefato que está sendo construído.
	\item Geralmente, os grandes projetos de softwares são diferentes dos projetos anteriores em algum aspecto. Somando-se a isto o fato da tecnologia em computadores evoluir muito rapidamente, mesmo gerentes de projetos com grande experiência podem achar difícil antecipar problemas e transferir lições aprendidas para novos projetos.
	\item Os processos de software são variáveis e de organização específica. Embora tenhamos tido grande progresso na padronização e melhorias dos processo, ainda não podemos dizer com certeza quando um processo irá levar a problemas de desenvolvimento, principalmente quando o projeto de software faz parte de um projeto de engenharia de sistemas mais amplo.
\end{alineas}


\subsection{Gerente de Projetos}
O gerente de projetos é a pessoa designada para liderar a equipe responsável por alcançar os objetivos do projeto \cite[p.~16]{pmi2013}. Segundo \citeonline[p.~485]{pressman2006} o Gerente de projetos é responsável por planejar, motivar, organizar e controlar os profissionais que fazem o trabalho de software.

Segundo o \citeonline{pmi2013} para que o gerente de projetos seja eficiente e eficaz, além de possuir habilidades na área específica do projeto em que está atuando, é desejável que também possua as seguintes competências:

\begin{alineas}
	\item conhecimento sobre gerenciamento de projetos;
	\item capacidade de fazer ou realizar algo quando aplica seus conhecimentos em gerenciamento de projetos ;
	\item comportamento gerencial e características de personalidade que fornecem a habilidade de guiar a equipe do projeto para se atingir os objetivos e equilibrar as restrições do projeto.
\end{alineas}

Ainda segundo o \citeonline{pmi2013}, o gerente de projetos deve possuir uma combinação equilibrada de habilidades éticas, interpessoais e conceituais para ajudá-los a analisar situações e agir de maneira apropriada:

\begin{alineas}
	\item liderança;
	\item construção de equipes;
	\item motivação;
	\item comunicação;
	\item influência;
	\item poder de decisão;
	\item consciência política e cultural;
	\item negociação;
	\item construção de confiança;
	\item gerenciamento de conflitos;
	\item \textit{coaching}.
\end{alineas}

Para \citeonline[p.~9]{kerzner2011}, dentre as habilidades desejadas para o Gerente de Projetos, as mais importantes não são as habilidades técnicas, mas sim as pessoais. O Gerente de projetos precisa conhecer sobre psicologia, comportamento humano e organizacional, relacionamento interpessoal e comunicação. Os Gerentes de projetos geralmente possuem grandes responsabilidades, mas pouca autoridade. Assim, precisam estar sempre negociando com a alta administração e gerências funcionais.

O trabalho do Gerente de projetos varia bastante, de acordo com a organização e o produto que está sendo desenvolvido. Mas para \citeonline{sommerville2011}, a maioria dos gerentes de projetos assumem, em algum momento, responsabilidade por algumas atividades bases comuns. O planejamento do projeto é uma destas atividades, onde o gerente deve garantir que o trabalho esteja sendo feito conforme os padrões e devendo acompanhar o progresso para garantir que o desenvolvimento está no prazo e dentro do orçamento. Também é responsável por fazer a ponte entre o desenvolvimento e os clientes e gerentes da empresa, mostrando o andamento do projeto de forma concisa e coerente, abstraindo as informações mais técnicas. Os gerenciamentos de riscos e pessoas são outras atividades desempenhadas em algum momento pelos gerentes de projetos. Eles devem ser capazes de avaliar e controlar os riscos que podem afetar o projeto e agir quando necessário, bem como estabelecer formas de trabalho que levem a um desempenho eficaz da equipe. Por fim, os gerentes devem ser capazes de elaborar propostas, descrevendo os objetivos do projeto e como ele será realizado, visando ganhar um contrato para executar um item de trabalho.

\subsection{Ciclo de vida do projeto}
Para compreender o gerenciamento de projetos é importante conhecer seu ciclo de vida. Segundo o \citeonline[p.~38]{pmi2013}, ciclo de vida do projeto é a serie de fases pelas quais o projeto passa, desde seu início até o término. As etapas entre a inicialização e a finalização do projeto podem ser sequenciais, iterativas ou sobrepostas e estas etapas são definidas ou moldadas conforme a necessidade de gerenciamento da organização. Independentemente dos moldes utilizados pelas empresas, todos os projetos possuem início e fim definidos e as fases podem ser desmembradas em entregas ou resultados intermediários.

Apesar de variar em tamanho e complexidade, todos os projetos podem ser mapeados para a estrutura de ciclo de vida a seguir \cite[p.~39]{pmi2013}:

\begin{alineas}
	\item início do projeto;
	\item organização e preparação;
	\item execução do trabalho do projeto;
	\item encerramento do projeto;
\end{alineas}

A estrutura acima e seu desenvolvimento ao londo do tempo pode ser ilustrada pela Figura~\ref{fig:ciclo_vida_projeto}, que mostra o nível de esforço e custo de cada fase.

\begin{figure}[htb]
\RawFloats
	\caption{\label{fig:ciclo_vida_projeto}Ciclo de vida de projetos (estrutura básica)}
	\begin{center}
	    \includegraphics[scale=0.50]{figuras/ciclo_vida_projeto.png}
	\end{center}
	\legend{Fonte: \cite[p.~13]{Cruz2013}}
\end{figure}

A partir desta estrutura básica, o projeto pode ser dividido em fases, que irão proporcionar benefícios ao trabalho que será feito para se atingir os objetivos do projeto. O ciclo de vida de projetos possuem cinco fases claras, segundo o PMBOK. Estas fases, também chamadas Grupos de Processos, são bem definidas e sequenciais, possuem passos a serem executados e são conhecidas por Iniciação, Planejamento, Execução, Monitoramento e Controle e Encerramento \cite[p.~14]{Cruz2013}. 

A Figura~\ref{fig:grupo_processos} mostra o esforço necessário em cada grupo de processo ao longo do ciclo de vida do projeto e a sobreposição entre os grupos.

\begin{figure}[htb]
\RawFloats
	\caption{\label{fig:grupo_processos}Interação entre os grupos de processos do PMBOK}
	\begin{center}
	    \includegraphics[scale=0.50]{figuras/grupos_processos.jpg}
	\end{center}
	\legend{Fonte: \cite[p.~51]{pmi2013}}
\end{figure}

\subsubsection{Iniciação}
A fase de inicialização do projeto é quando a autorização para o início de um novo projeto, ou nova fase de um projeto é obtida, o escopo inicial é definido, os recursos financeiros iniciais são comprometidos, os interessados internos e externos são identificados e o Gerente de projeto é definido. O objetivo principal desta fase é alinhar as expectativas das partes interessadas com os objetivos do projeto, dar-lhes visibilidade sobre o escopo e objetivos, e mostrar como a sua participação no projeto irá contribuir para que as expectativas sejam atingidas \cite[p.~54]{pmi2013}.

\subsubsection{Planejamento}
Neste grupo de processos estão os processos realizados para estabelecer o escopo, definir e refinar objetivos e desenvolver o curso de ações necessário para se atingir estes objetivos. O principal benefício deste grupo de processos é delinear a estratégia e a tática, e também o caminho para a conclusão do projeto com sucesso. O planejamento explorará as seguintes áreas apresentadas no Guia PMBOK: escopo, tempo, qualidade, comunicações, recursos humanos, riscos, aquisições e gerenciamento das partes interessadas \cite[p.~55]{pmi2013}.

\subsubsection{Execução}
Este grupo consiste dos processos executados para se concluir o trabalho definido no plano de gerenciamento do projeto. Este grupo também envolve coordenar a equipe e recursos, e gerenciar as expectativas das partes interessadas. Durante a execução do projeto podem surgir alterações com relação ao planejado e podem surgir riscos não esperados. Estas alterações são analisadas e, se necessário, o plano se de gerenciamento é alterado \cite[p.~55]{pmi2013}.

\subsubsection{Monitoramento e Controle}
O grupo de monitoramento e controle consiste dos processos necessários para acompanhar e analisar o andamento e o desempenho do projeto. O principal benefício deste grupo é a medição e análise do desempenho do projeto com o objetivo de identificar variações no plano de gerenciamento do projeto. Este controle fornece à equipe uma visão melhor do andamento do projeto e identifica as áreas que precisam de mais atenção \cite[p.~55]{pmi2013}.

\subsubsection{Encerramento}
Os processos deste grupo tem a finalidade de encerrar todas as atividades de todos os grupos de processos de gerenciamento do projeto, assim finalizado formalmente o projeto, as fases ou as obrigações contratuais. Este grupo de processo verifica se os processos definidos estão completos em todos os grupos de processos. Este grupo também formaliza encerramentos prematuros do projeto, como por exemplo, projetos abortados \cite[p.~55]{pmi2013}.


% ----------------------------------------------------------
% ELEMENTOS PÓS-TEXTUAIS
% ----------------------------------------------------------
\postextual
% ----------------------------------------------------------

% ----------------------------------------------------------
% Referências bibliográficas
% ----------------------------------------------------------
\bibliography{references}


% ----------------------------------------------------------
% Glossário
% ----------------------------------------------------------
%
% Consulte o manual da classe abntex2 para orientações sobre o glossário.
%
%\glossary

% ----------------------------------------------------------
% Apêndices
% ----------------------------------------------------------

% ---
% Inicia os apêndices
% ---

% ---


% ----------------------------------------------------------
% Anexos
% ----------------------------------------------------------

% ---
% Inicia os anexos
% ---


%---------------------------------------------------------------------
% INDICE REMISSIVO
%---------------------------------------------------------------------
\phantompart
\printindex
%---------------------------------------------------------------------

\end{document}