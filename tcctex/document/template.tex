%% Documento gerado com FastFormat. Copyright 2015 http://www.fastformat.co/

\documentclass[
    % -- opções da classe memoir --
    12pt,               % tamanho da fonte
    openright,          % capítulos começam em pág ímpar (insere página vazia caso preciso)
    {{ print_side }},            % para impressão em verso e anverso. Oposto a % oneside
    {{ paper }},            % tamanho do papel. 
    % -- opções da classe abntex2 --
    %chapter=TITLE,     % títulos de capítulos convertidos em letras maiúsculas
    %section=TITLE,     % títulos de seções convertidos em letras maiúsculas
    %subsection=TITLE,  % títulos de subseções convertidos em letras maiúsculas
    %subsubsection=TITLE,% títulos de subsubseções convertidos em letras maiúsculas
    % -- opções do pacote babel --
    english,            % idioma adicional para hifenização
    french,             % idioma adicional para hifenização
    spanish,            % idioma adicional para hifenização
    {{ language }}              % o último idioma é o principal do documento
    ]{abntex2}

% ---
% Pacotes básicos 
% ---




\usepackage{lmodern}            % Usa a fonte Latin Modern          
\usepackage[T1]{fontenc}        % Selecao de codigos de fonte.
\usepackage[utf8]{inputenc}     % Codificacao do documento (conversão automática dos acentos)
\usepackage{lastpage}           % Usado pela Ficha catalográfica
\usepackage{indentfirst}        % Indenta o primeiro parágrafo de cada seção.
\usepackage{color,xcolor}              % Controle das cores
\usepackage{graphicx}           % Inclusão de gráficos
\usepackage{microtype}          % para melhorias de justificação
\usepackage{booktabs}
\usepackage{tabulary}
\usepackage{caption}
\usepackage{soul}
\usepackage{rotating}



% ---

% ---
% Pacotes adicionais, usados apenas no âmbito do Modelo Canônico do abnteX2
% ---
\usepackage{lipsum}             % para geração de dummy text
% ---

\usepackage{abnt_customization}



% ---
% Informações de dados para CAPA e FOLHA DE ROSTO
% ---
\titulo{%
{{ title|default:'Título do Documento' }}%
}%

\autor{ {{ a.name_escaped }}\\Autor do Documento }
\local{ {{ opt_text_local|default:'LOCAL' }} }
\data{ {{ opt_text_data|default:'DATA' }} }
\orientador{ {{ opt_text_orientador|default:'Nome do Orientador' }} }
\coorientador{ {{ opt_text_coorientador|default:'Co-orientador'}} }
\instituicao{%
  \bfseries
  \MakeUppercase{ {{opt_text_universidade|default:'Nome da Instituição'}} }
  \par
  \MakeUppercase{ {{ opt_text_centro|default:'' }} }
  \par
  \MakeUppercase{ {{ opt_text_departamento|default:'' }} }
  \par
  \MakeUppercase{ {{ opt_text_programa|default:'' }} } }
\tipotrabalho{ {{ opt_text_tipo_de_trabalho|default:'Dissertação de Mestrado' }} }
% O preambulo deve conter o tipo do trabalho, o objetivo, 
% o nome da instituição e a área de concentração 
\preambulo{ {{ opt_text_preambulo|default:'O texto do Pré-âmbulo ainda está vazio. Preencha-o nos atributos do documento. O texto do Pré-âmbulo ainda está vazio. Preencha-o nos atributos do documento. O texto do Pré-âmbulo ainda está vazio. Preencha-o nos atributos do documento.' }} }
% ---

\def\volume{ {{ opt_text_volume }} }


% --- 
% Espaçamentos entre linhas e parágrafos 
% --- 

% O tamanho do parágrafo é dado por:
\setlength{\parindent}{1.3cm}

% Controle do espaçamento entre um parágrafo e outro:
\setlength{\parskip}{0.2cm}  % tente também \onelineskip

% ---
% compila o indice
% ---
\makeindex
% ---

% ----
% Início do documento
% ----
\begin{document}

% Seleciona o idioma do documento (conforme pacotes do babel)
%\selectlanguage{english}
%\selectlanguage{brazil}

% Retira espaço extra obsoleto entre as frases.
\frenchspacing 



% ----------------------------------------------------------
% ELEMENTOS TEXTUAIS
% ----------------------------------------------------------
\textual

{{ content }}

% ----------------------------------------------------------
% ELEMENTOS PÓS-TEXTUAIS
% ----------------------------------------------------------
\postextual
% ----------------------------------------------------------

% ----------------------------------------------------------
% Referências bibliográficas
% ----------------------------------------------------------


\bibliography{references}


% ----------------------------------------------------------
% Glossário
% ----------------------------------------------------------
%
% Consulte o manual da classe abntex2 para orientações sobre o glossário.
%
%\glossary

% ----------------------------------------------------------
% Apêndices
% ----------------------------------------------------------

% ---
% Inicia os apêndices
% ---

\begin{apendicesenv}

% Imprime uma página indicando o início dos apêndices
\partapendices

{{ appendix }}

\end{apendicesenv}

% ---


% ----------------------------------------------------------
% Anexos
% ----------------------------------------------------------

% ---
% Inicia os anexos
% ---

\begin{anexosenv}

% Imprime uma página indicando o início dos anexos
\partanexos

{{ attachment }}

\end{anexosenv}


%---------------------------------------------------------------------
% INDICE REMISSIVO
%---------------------------------------------------------------------
\phantompart
\printindex
%---------------------------------------------------------------------

\end{document}